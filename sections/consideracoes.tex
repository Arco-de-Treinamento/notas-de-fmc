\section*{Considerações iniciais}

Inicialmente as notas de jubilado foram idealizadas como uma forma de exercitar a escrita em LaTeX do autor e, simultaneamente, estudar a matemática discreta, utilizando a metodologia \abreaspas Learn In Public\fechaaspas, uma modalidade de estudo onde ocorre a produção ativa de conteúdo, ao invés do simples consumo passivo e desesperado de todo e qualquer material disponível na esperança de absorver um tostão de conteúdo.  

Este projeto também visa responder algumas perguntas recorrentes em círculos de conversa dos recém-iniciados no fantástico mundo do BTI, como \abreaspas pra quê estou estudando isso?\fechaaspas ou \abreaspas cadê o código?\fechaaspas. Sendo assim, a cada capítulo também existirá um tópico referente a uma aplicação do tema na ciência da computação.

É importante lembrar que, assim como você, caro leitor curioso, o autor também se encontra no estado de \abreaspas estudante lascado\fechaaspas. Por isso, o desenvolvimento do material seguirá o insano ritmo de aprendizado de um universitário brasileiro.

Mais uma vez, vale ressaltar que o presente material não foi construído por professores ou qualquer outro indivíduo dotado de um título no campo da matemática, sendo unicamente uma paródia das notas de aula oficiais da disciplina de Matemática Elementar (IMD1001). O principal intuito desse documento é descontrair durante os estudos. Em hipótese nenhuma utilize esse artigo como única fonte de conhecimento.