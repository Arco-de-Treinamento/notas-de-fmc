\section{Complementar}
\label{sec:complementar}

Podemos determinar o \emph{conjunto complementar} como o conjunto \textbf{formado pelos elementos que não estão presentes em um determinado conjunto, mas que pertencem ao universo}. De certo modo, dado um conjunto $A$, o \emph{conjunto complementar} de $A$ seria a diferença entre o conjunto $A$ e o conjunto $\U$. 

O \emph{conjunto complementar} também pode ser definido da seguinte forma:

\[
	A\complementar = \conjunto{ x \taisque x \naopertence A}.
\]

\importitem{complementar-prop}
\importitem{demorgan}

\begin{remark}
    A demonstração das propriedades anteriores pode ser vista nas Notas de Aula da disciplina de Matemática Elementar\cite[pp. 9 - 10]{notas-de-aula}.
\end{remark}