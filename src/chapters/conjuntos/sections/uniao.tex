\subsection{União}

A \emph{união} de dois ou mais conjuntos pode ser descrita como o \textbf{conjunto} formado pelos elementos que pertencem a pelo menos um dos cojuntos. Matematicamente, a união é representada pelo símbolo \entreaspas{$\uniao$}. Desse modo, se temos um conjunto $A$ e um outro $B$, a união pode ser dada por: 

\[
    A \uniao B = \conjunto{ x \tq x \em A \ou x \em B}.
\]


\begin{remark}
    Note que $x$ é elemento de $A \uniao B$ apenas se $x \pertence A$ ou se $x \pertence B$.
\end{remark}