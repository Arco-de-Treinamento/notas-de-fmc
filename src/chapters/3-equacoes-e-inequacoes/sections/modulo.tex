\section{Módulos}
O conceito de \emph{módulo} ou \emph{valor absoluto} pode ser compreendido como uma função que transforma um valor em sua magnitude. O \emph{módulo} de um número real $x$, representado por $\modu x$, é definido por:

$$
\modu x =
\begin{cases}
x , & \text{se $x \ge 0$} \\
-x, & \text{se $x<0$}.
\end{cases}
$$
\\

Na resolução de equações ou desigualdades que envolvem \emph{módulo} é necessário analisar separadamente cada possibilidade de sinal na expressão dentro do módulo. 
\\

Considere a equação $\modu{2x - 5} = 3$. Note que:

\begin{align*}
    \modu{2x-5} = 
    \begin{cases}
    2x-5,    & \text{se } 2x-5 \ge 0 \iff x \ge 5/2 \\
    -(2x-5), & \text{se } 2x-5 < 0 \iff x < 5/2
    \end{cases}
\end{align*}

Se $x \ge 5/2$, teremos:

\begin{align*}
\modu{2x-5}=3 &\iff 2x-5=3 \\
              &\iff x=4
\end{align*}

Como $x=4\ge5/2$, temos que $4 \in S$. Se $x<5/2$, teremos:

\begin{align*}
    \modu{2x-5}=3 &\iff -\prn{2x-5} = 3 \\
                  &\iff -2x = -2 \\
                  &\iff x = 1 
\end{align*}

Como $x=1<5/2$, então $1 \in S$. Das análises dos dois casos, concluímos que o conjunto solução é $S=\set{1,4}$.
