\subsection{diferença}

Chama-se \emph{interseção} a operação que retorna um conjunto formado pelos elementos que estão simultaneamente presentes em ambos os conjuntos $A$ e $B$. Em notação matemática a interseção é representada por \entreaspas{$\inter$} e também pode ser definido por:  

Dado dois conjunto $A$ e $B$, a \emph{diferença} é o conjunto que contém os elementos de $A$ que não estão contidos em $B$. Kenneth H. Rosen ainda define a \emph{diferença} como \emph{o complemento de $B$ em relação a $A$} \cite[pp. 123]{kenneth2010}. Posteriormente abordaremos a definição de complemento de um conjunto.

\[
	A \diferenca B = \conjunto{ x \talque  x \pertence A \e x \naopertence B}.
\]
