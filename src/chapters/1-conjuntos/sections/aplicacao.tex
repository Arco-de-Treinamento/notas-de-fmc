\section{Aplicação da teoria dos conjuntos na computação}

A teoria dos conjuntos possui inúmeras aplicações no desenvolvimento de software, especialmente em linguagens que trabalham com programação orientada a objetos. Um exemplo comum é a utilização dos \textbf{sets} para manipulação de dados.

Em Python, por exemplo, é possível utilizar os sets para criar conjuntos de dados conforme o necessário. Assim como os conjuntos matemáticos, os sets são definidos como um tipo de dado desordenado que não possui elementos duplicados.\\

\begin{codesnip}{Exemplo em Python}{python}
# Cria uma lista com elementos repetidos
lista = [1, 2, 3, 2, 4, 3, 5, 6, 5]

# Converte a lista em um conjunto para remover elementos repetidos
conjunto = set(lista)

# Imprime o conjunto
print(conjunto)
# Output: {1, 2, 3, 4, 5, 6}
\end{codesnip}

Neste exemplo, a classe "set" é usada para criar um conjunto a partir da lista. O conjunto remove automaticamente quaisquer elementos repetidos e mantém somente os elementos únicos. Além disso, a classe set possibilita operações como união, interseção, diferença entre conjuntos, subconjunto e etc.

Outra aplicação é a utilização de conjuntos para verificar a pertença de um elemento a um conjunto, é muito mais rápido do que percorrer uma lista ou dicionário verificando a existencia de um elemento.\\

Para mais informações sobre a utilização de sets em Python acesse: \link{https://pythonacademy.com.br/blog/sets-no-python}{Sets no Python}. \\




