\section{Introdução}

Inicialmente, um \emph{conjunto} pode ser considerado como uma coleção de elementos quaisquer. Dessa forma, um conjunto é definido unicamente por seus membros, que podem ser de qualquer tipo, incluindo outros conjuntos. 

Temos ainda a definição trazida por Kenneth H. Rosen, em seu livro “Matemática Discreta e Suas Aplicações”, que sugere o \emph{conjunto} como uma coleção não ordenada de objetos onde os objetos nele contidos não são chamados de "membros do conjunto" e sim “elementos pertencentes ao conjunto”, ressaltando a característica de pertencimento \cite[pp. 111--112]{kenneth2010}.

Em notação matemática, um \emph{conjunto} pode ser representado por uma lista de elementos entre chaves, como, por exemplo, o conjunto $A = \conjunto {\texttt 1, \texttt 2, \texttt 4}$, formado pelos números inteiros 1, 2 e 4. Outro método que pode ser utilizado é a definição por sua lei de formação, como o conjunto dos números inteiros $\inteiros = \conjunto {x \mid x \in \reais \text{ e } x = \modulo{x} }$, que pode ser entendido como o conjunto de todos os elementos $x$'s tal que $x$ é igual ao módulo de $x$.

Um conjunto pode ainda ser descrito verbalmente por sentenças que definem seus elementos. Para esse caso podemos definir um conjunto $A$ como \emph{o conjunto dos números pares menores que 10}, compostos pelos números 2, 4, 6 e 8. 

Nesta seção, além de definir suas propriedades, também estudaremos as operações e aplicações diretas que os \emph{conjuntos} podem possuir na computação.