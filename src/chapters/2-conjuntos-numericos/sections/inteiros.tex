\subsection{Conjunto dos números inteiros}

Ao conjunto de todos os números inteiros, incluindo os números positivos e negativos, tal que $A = \conjunto{\etc , -m -1, -m, \etc, -1, 0, 1,  \etc , n, n+1, \etc}$, damos o nome de \emph{conjunto dos números inteiros}, representado por $\inteiros$. 

Em casos onde desejamos trabalhar com algumas variações desse conjunto utilizamos as seguintes notações:

\begin{align*}
	\inteiros\naonulos     &= \inteiros \menos \unitario{0}              &&\text{ (inteiros não nulos);}     \\
	\inteiros\naonegativos &= \naturais                                  &&\text{ (inteiros não negativos);} \\
	\inteiros\positivos    &= \naturais\naonulos                         &&\text{ (inteiros positivos);}     \\
	\inteiros\naopositivos &= \conjunto{\etc, -{m-1}, -m, \etc, -1, 0}   &&\text{ (inteiros não positivos);} \\
	\inteiros\negativos    &= \inteiros\naopositivos \menos \unitario{0} &&\text{ (inteiros negativos).}     
\end{align*}