\section{Teorema da Indução FInita}

O \emph{princípio da indução finita} afirma que se uma afirmação é verdadeira para o \textbf{número 1}, então ela também será verdadeira para o número seguinte, \textbf{k + 1}. Logo, a afirmação é verdadeira para todos os números naturais maiores ou iguais a 1. 

Sendo assim, o \emph{princípio da indução finita} pode ser formulado como:

\begin{enumerate}[(a)]
    \item $p \prn{k_0}$ é verdadeira;
    \item Se $p \prn k$ é verdadeira, então $p \prn {k+1}$ também
    é verdadeira, para todo $k \geq k_0$.
\end{enumerate}
  
Portanto, a afirmação $p \prn k$ é verdadeira para todo número natural k.

\begin{remark}
    Podemos entender afirmação (a) como a \emph{base da indução}, e a (b) como \emph{passo indutivo}. O fato de que $p \prn n$ é verdadeira no item (b) é chamado de \emph{hipótese de indução}.
\end{remark}

\importitem{inducao-finita-forte}
