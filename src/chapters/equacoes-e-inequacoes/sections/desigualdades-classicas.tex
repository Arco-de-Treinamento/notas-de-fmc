\section{Desigualdades Clássicas}

As \emph{desigualdades clássicas} são um conjunto de desigualdades que surgem com frequência em diferentes campos da matemática. Nesse tópico teremos uma breve descrição de algumas delas. Além das já conhecidas $\modu a \ge 0$ e $a^2 \ge 0$, que podem ser compreendidas pela definição de módulo, também possuímos as seguintes inequações: 

\begin{theorem}
    Desigualdade triangular:
    \begin{equation*}
        \modu {a+b} \le \modu a + \modu b
    \end{equation*}
\end{theorem}

\begin{theorem}
    Para quaisquer $x, y \in \R$, vale:

    \begin{equation*}
        xy \le \frac {x^2 +y^2} 2
    \end{equation*}

    Além disso, a igualdade acontece se, e somente se, $x=y$.
\end{theorem}

\begin{theorem}[Desigualdade das médias aritmética e geométrica]
    Para quaisquer $n \in \nnats$ e $a_1, a_2, \dots , a_n \in \R_+$, vale:
    
    \begin{equation*}
        \sqrt[n]{a_1\dots a_n} \leq \frac {a_1 + \dots + a_n} n
    \end{equation*}
\end{theorem}

\begin{theorem}[Desigualdade das médias harmônica e geométrica]
    Para quaisquer $n \in \nnats$ e $a_1, a_2, \dots , a_n \in \R_+^*$, vale:
    
    \begin{equation*}
        \frac n {\frac 1 {a_1} + \dots + \frac 1 {a_n}}  \leq \sqrt[n]{a_1\dots a_n}  
    \end{equation*}
\end{theorem}

\begin{theorem}[Desigualdade de Cauchy-Schwarz]
    Sejam $x_1, \dots , x_n, y_1, \dots y_n \in \R$. O seguinte vale:
    
    \begin{equation*}
        \modu{x_1y_1 + \dots + x_ny_n} \leq \sqrt{x^2_1+ \dots + x^2_n}
        \cdot \sqrt{y^2_1+ \dots + y^2_n}.
    \end{equation*}

    Além disso, a igualdade só ocorre se existir um número real $\alpha$ tal que $x_1 = \alpha y_1$, ..., $x_n = \alpha y_n$.
\end{theorem}