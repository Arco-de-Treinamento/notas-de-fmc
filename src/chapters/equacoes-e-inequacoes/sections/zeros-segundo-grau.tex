\subsection{Zeros ou raízes da função do 2° grau}

As raízes da \emph{função do segundo grau} são valores de $x$ tais que $f(x) = 0$ e $x \nos \reais$. Manipulando a sua forma canônica, podemos determinar as raízes como:

\begin{align*}
    ax^2 + bx + c = 0 & \sse a \abs{\prn{x + \frac{b}{2a}}^2 - \frac{\Delta}{4a^2} } = 0 \\
                      & \sse \prn{x + \frac{b}{2a}}^2 - \frac{\Delta}{4a^2} = 0 \\
                      & \sse \prn{x + \frac{b}{2a}}^2 = \frac{\Delta}{4a^2} \\
                      & \sse x + \frac{b}{2a} = \pm \frac{\sqrt{\Delta}}{2a} \\
                      & \sse x = \frac{ - b \pm \sqrt{\Delta}}{2a}
\end{align*}

Além disso, temos que o \emph{discriminante} da equação $\Delta$ pode ser definido como $b^2 - 4ac$, de forma que:

\begin{itemize}
    \item Se $\Delta > 0$, existem duas soluções reais;
	\item Se $\Delta = 0 $, existe uma solução real ($x_1 = x_2 = \frac{-b}{2a})$;
	\item Se $\Delta < 0$, não existe solução real.
\end{itemize}