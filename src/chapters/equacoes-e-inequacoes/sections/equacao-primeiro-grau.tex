\section{Equação do 1° grau}

Uma \emph{equação do primeiro grau} representa uma igualdade entre duas expressões matemáticas, na qual existe uma incógnita que pode ser isolada e calculada. Normalmente a \emph{equação do prmeiro grau} pode ser representada da seguinte fora:

\begin{align*}
    ax+b=0
\end{align*}

Onde o $x$ é a incógnita, um numero real a ser encontrado, e $a,b\in \R$ são constantes. Além disso, temos que $a \neq 0$.


\begin{proposition}[Propriedades]
    \label{prop:props-eq}
    Sejam $a, b, c, d \in \R$. Temos que:
    \begin{enumerate}
        \item 
        \label{eq:soma} 
        Se $a = b$ e $c = d$, então $a + c = b + d$. Em particular, vale a implicação $a=b \implies a+c = b+c$;
        \item 
        \label{eq:multiplicacao}
        Se $a = b$ e $c = d$, então $a \cdot c = b \cdot d$. Nesse caso, vale a  implicação $a=b \implies ac = bc$.
    \end{enumerate}
\end{proposition}