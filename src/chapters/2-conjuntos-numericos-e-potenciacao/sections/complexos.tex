\subsection{Conjunto dos números complexos}

O \emph{conjunto dos números complexos} pode ser compreendido como o conjunto que inclui tanto números reais quanto imaginários, tal que $i = \raiz{-1}$ é um \emph{número imaginário}. O \emph{conjunto dos números complexos} é representado por  $\C = \conjunto{ a+bi \taisque a,b \pertence \R}$ e cada número complexo é formado por uma \textbf{parte real} e uma \textbf{parte imaginária}, escritas na forma $a + bi$, onde a é \textbf{a} parte real e \textbf{b} é a parte imaginária. 

Além disso, ainda temos o \textbf{conjugado} do número complexo, obtido ao trocar o sinal do seu componente imaginário, tal que, se $c = a + bi$ é um número complexo, sua forma conjugada pode ser representada por $c* = a - bi$.