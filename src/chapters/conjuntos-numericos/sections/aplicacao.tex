\section{Aplicação da conjuntos muméricos na computação}

Os \emph{conjuntos numéricos} possuem ampla utilização em diversas áreas da programação, incluindo a ciência de dados, calculo numéricos e álgebra linear, sendo utilizados para representar vetores, matrizes ou em equações. No campo da inteligência artificial, os \emph{conjuntos numéricos} ainda podem ser utilizados no aprendizado de máquina e na representação dos dados para treinamento de modelos.

Outra importante aplicação está na geometria computacional. Em C, por exemplo, podemos utilizar os conjuntos para calcular a distância entre pontos. \\

\begin{codesnip}{Exemplo em C}{C}
double x1, y1, x2, y2, distance;

// recebe a posicao do ponto 1
printf("Informe as coordenadas do ponto 1: x1, y1 :\n");
scanf("%lf,%lf", &x1, &y1);

// recebe a posicao do ponto 2
printf("Informe as coordenadas do ponto 2: x2, y2:\n");
scanf("%lf,%lf", &x2, &y2);

// fórmula de Pitágoras
distance = sqrt(pow((x2 - x1), 2) + pow((y2 - y1), 2));

// Imprimindo o resultado
printf("A distância entre os pontos é %lf\n", distance);
\end{codesnip}

Neste exemplo, utilizamos as funções \textbf{pow} e \textbf{sqrt}, respectivamente utilizadas no cálculo de potência e raiz quadrada, da biblioteca \textbf{math.h}, para calcular a distância entre o primeiro e o segundo ponto através da fórmula de Pitágoras.

Para mais informações sobre a biblioteca math.c acesse: \link{http://linguagemc.com.br/a-biblioteca-math-h/}{A biblioteca math.h}. \\




